% A LaTeX template for MSc Thesis submissions to 
% Politecnico di Milano (PoliMi) - School of Industrial and Information Engineering
%
% S. Bonetti, A. Gruttadauria, G. Mescolini, A. Zingaro
% e-mail: template-tesi-ingind@polimi.it
%
% Last Revision: October 2021
%
% Copyright 2021 Politecnico di Milano, Italy. NC-BY

\documentclass{Configuration_Files/PoliMi3i_thesis}
\usepackage{algorithm}
%------------------------------------------------------------------------------
%	REQUIRED PACKAGES AND  CONFIGURATIONS
%------------------------------------------------------------------------------

% CONFIGURATIONS
\usepackage{parskip} % For paragraph layout
\usepackage{setspace} % For using single or double spacing
\usepackage{emptypage} % To insert empty pages
\usepackage{multicol} % To write in multiple columns (executive summary)
\setlength\columnsep{15pt} % Column separation in executive summary
\setlength\parindent{0pt} % Indentation
\raggedbottom  

% PACKAGES FOR TITLES
\usepackage{titlesec}
% \titlespacing{\section}{left spacing}{before spacing}{after spacing}
\titlespacing{\section}{0pt}{3.3ex}{2ex}
\titlespacing{\subsection}{0pt}{3.3ex}{1.65ex}
\titlespacing{\subsubsection}{0pt}{3.3ex}{1ex}
\usepackage{color}

% PACKAGES FOR LANGUAGE AND FONT
\usepackage[english]{babel} % The document is in English  
\usepackage[utf8]{inputenc} % UTF8 encoding
\usepackage[T1]{fontenc} % Font encoding
\usepackage[11pt]{moresize} % Big fonts

% PACKAGES FOR IMAGES
\usepackage{graphicx}
\usepackage{transparent} % Enables transparent images
\usepackage{eso-pic} % For the background picture on the title page
\usepackage{subfig} % Numbered and caption subfigures using \subfloat.
\usepackage{tikz} % A package for high-quality hand-made figures.
\usetikzlibrary{}
\graphicspath{{./Images/}} % Directory of the images
\usepackage{caption} % Coloured captions
\usepackage{xcolor} % Coloured captions
\usepackage{amsthm,thmtools,xcolor} % Coloured "Theorem"
\usepackage{float}

% STANDARD MATH PACKAGES
\usepackage{amsmath}
\usepackage{amsthm}
\usepackage{amssymb}
\usepackage{amsfonts}
\usepackage{bm}
\usepackage[overload]{empheq} % For braced-style systems of equations.
\usepackage{fix-cm} % To override original LaTeX restrictions on sizes

% PACKAGES FOR TABLES
\usepackage{tabularx}
\usepackage{longtable} % Tables that can span several pages
\usepackage{colortbl}

% PACKAGES FOR ALGORITHMS (PSEUDO-CODE)
\usepackage{algorithm}
\usepackage{algorithmic}

% PACKAGES FOR REFERENCES & BIBLIOGRAPHY
\usepackage[colorlinks=true,linkcolor=black,anchorcolor=black,citecolor=black,filecolor=black,menucolor=black,runcolor=black,urlcolor=black]{hyperref} % Adds clickable links at references
\usepackage{cleveref}
\usepackage[square, numbers, sort&compress]{natbib} % Square brackets, citing references with numbers, citations sorted by appearance in the text and compressed
\bibliographystyle{abbrvnat} % You may use a different style adapted to your field

% OTHER PACKAGES
\usepackage{pdfpages} % To include a pdf file
\usepackage{afterpage}
\usepackage{lipsum} % DUMMY PACKAGE
\usepackage{fancyhdr} % For the headers
\fancyhf{}

% Input of configuration file. Do not change config.tex file unless you really know what you are doing. 
% Define blue color typical of polimi
\definecolor{bluepoli}{cmyk}{0.4,0.1,0,0.4}

% Custom theorem environments
\declaretheoremstyle[
  headfont=\color{bluepoli}\normalfont\bfseries,
  bodyfont=\color{black}\normalfont\itshape,
]{colored}

% Set-up caption colors
\captionsetup[figure]{labelfont={color=bluepoli}} % Set colour of the captions
\captionsetup[table]{labelfont={color=bluepoli}} % Set colour of the captions
\captionsetup[algorithm]{labelfont={color=bluepoli}} % Set colour of the captions

\theoremstyle{colored}
\newtheorem{theorem}{Theorem}[chapter]
\newtheorem{proposition}{Proposition}[chapter]

% Enhances the features of the standard "table" and "tabular" environments.
\newcommand\T{\rule{0pt}{2.6ex}}
\newcommand\B{\rule[-1.2ex]{0pt}{0pt}}

% Pseudo-code algorithm descriptions.
\newcounter{algsubstate}
\renewcommand{\thealgsubstate}{\alph{algsubstate}}
\newenvironment{algsubstates}
  {\setcounter{algsubstate}{0}%
   \renewcommand{\STATE}{%
     \stepcounter{algsubstate}%
     \Statex {\small\thealgsubstate:}\space}}
  {}

% New font size
\newcommand\numfontsize{\@setfontsize\Huge{200}{60}}

% Title format: chapter
\titleformat{\chapter}[hang]{
\fontsize{50}{20}\selectfont\bfseries\filright}{\textcolor{bluepoli} \thechapter\hsp\hspace{2mm}\textcolor{bluepoli}{|   }\hsp}{0pt}{\huge\bfseries \textcolor{bluepoli}
}

% Title format: section
\titleformat{\section}
{\color{bluepoli}\normalfont\Large\bfseries}
{\color{bluepoli}\thesection.}{1em}{}

% Title format: subsection
\titleformat{\subsection}
{\color{bluepoli}\normalfont\large\bfseries}
{\color{bluepoli}\thesubsection.}{1em}{}

% Title format: subsubsection
\titleformat{\subsubsection}
{\color{bluepoli}\normalfont\large\bfseries}
{\color{bluepoli}\thesubsubsection.}{1em}{}

% Shortening for setting no horizontal-spacing
\newcommand{\hsp}{\hspace{0pt}}

\makeatletter
% Renewcommand: cleardoublepage including the background pic
\renewcommand*\cleardoublepage{%
  \clearpage\if@twoside\ifodd\c@page\else
  \if@twocolumn\hbox{}\newpage\fi\fi\fi}
\makeatother

%For correctly numbering algorithms
\numberwithin{algorithm}{chapter}

%----------------------------------------------------------------------------
%	NEW COMMANDS DEFINED
%----------------------------------------------------------------------------
\usepackage{hyperref}
\hypersetup{
    colorlinks = true,
    urlcolor=blue,
}
\usepackage{listings}  % Package for code formatting
\usepackage{xcolor}    % For color customization

\lstdefinestyle{pythonStyle}{
    language=Python,
    basicstyle=\ttfamily\footnotesize,
    keywordstyle=\color{blue},
    stringstyle=\color{red},
    commentstyle=\color{gray},
    showstringspaces=false,
    frame=single,
    breaklines=true
}

% EXAMPLES OF NEW COMMANDS
\newcommand{\bea}{\begin{eqnarray}} % Shortcut for equation arrays
\newcommand{\eea}{\end{eqnarray}}
\newcommand{\e}[1]{\times 10^{#1}}  % Powers of 10 notation

%----------------------------------------------------------------------------
%	BEGIN OF YOUR DOCUMENT
%----------------------------------------------------------------------------

\begin{document}

\fancypagestyle{plain}{%
\fancyhf{} % Clear all header and footer fields
\fancyhead[RO,RE]{\thepage} %RO=right odd, RE=right even
\renewcommand{\headrulewidth}{0pt}
\renewcommand{\footrulewidth}{0pt}}

%----------------------------------------------------------------------------
%	TITLE PAGE
%----------------------------------------------------------------------------

\pagestyle{empty} % No page numbers
\frontmatter % Use roman page numbering style (i, ii, iii, iv...) for the preamble pages

\puttitle{
	title=IOT Challenge 2 - Exercise, % Title of the thesis
	name=Matteo Leonardo Luraghi, % Author Name and Surname
	course=, % Study Programme (in Italian)
	ID  = 10772886,  % Student ID number (numero di matricola)
	academicyear={2024-25},  % Academic Year
} % These info will be put into your Title page 

%----------------------------------------------------------------------------
%	PREAMBLE PAGES: ABSTRACT (inglese e italiano), EXECUTIVE SUMMARY
%----------------------------------------------------------------------------
\startpreamble
\setcounter{page}{1} % Set page counter to 1

%----------------------------------------------------------------------------
%	LIST OF CONTENTS/FIGURES/TABLES/SYMBOLS
%----------------------------------------------------------------------------

% TABLE OF CONTENTS
\thispagestyle{empty}
\tableofcontents % Table of contents 
\thispagestyle{empty}
\cleardoublepage

%-------------------------------------------------------------------------
%	THESIS MAIN TEXT
%-------------------------------------------------------------------------
% In the main text of your thesis you can write the chapters in two different ways:
%
%(1) As presented in this template you can write:
%    \chapter{Title of the chapter}
%    *body of the chapter*
%
%(2) You can write your chapter in a separated .tex file and then include it in the main file with the following command:
%    \chapter{Title of the chapter}
%    \input{chapter_file.tex}
%
% Especially for long thesis, we recommend you the second option.

\addtocontents{toc}{\vspace{2em}} % Add a gap in the Contents, for aesthetics
\mainmatter % Begin numeric (1,2,3...) page numbering

% --------------------------------------------------------------------------
% NUMBERED CHAPTERS % Regular chapters following
% --------------------------------------------------------------------------

\cleardoublepage

\chapter{EQ1}

\section{EQ1.a}
In this scenario, the temperature sensor acts as the CoAP server, as it holds the temperature measurement resource.

The valve initiates communication by sending a GET request to the sensor, enabling observe mode. This allows the sensor to automatically push updated temperature values to the valve every 5 minutes, without requiring additional requests from the valve.

The GET request is sent as a non-confirmable message to minimize energy consumption, which is feasible given the fact that the Wi-Fi network is ideal.

\subsection{Energy Consumption of the Valve}
To compute the energy $E_V$ used by the valve, we take into account:
\begin{itemize}
    \item the transmission of the first GET request ($60B$), where the observation mode is set
    \item the receptions of the GET responses ($55B$) from the valve, that happen every 5 minutes: $60/5=12$ receptions per hour
    \item the 2 processing per hour of the average temperature (1 every 30 minutes)
\end{itemize}

\begin{equation}
    E_{V} = E_{TX}*60*8  [nJ] + 12*24*E_{RX}*55*8  [nJ] + 2*24*E_c [mJ] = 122.57 [mJ]
\end{equation}

\newpage
\subsection{Energy Consumption of the Temperature Sensor}
To compute the energy $E_S$ of the temperature sensor, we take into account:
\begin{itemize}
    \item the reception of the first GET request ($60B$)
    \item the 12 GET response transmissions per hour containing the new temperature value ($55B$)
\end{itemize}
\begin{equation}
    E_S = E_{RX}*60*8  [nJ] + 12*24*E_{TX}*55*8  [nJ] = 6.36 [mJ]
\end{equation}

\subsection{Total Energy Consumption}
The total energy consumed by the valve and the temperature sensor is given by:
\begin{equation}
    E = E_V + E_T = 128.93 [mJ]
\end{equation}

\newpage
\section{EQ1.b}
In this scenario, the Raspberry Pi acts as an MQTT broker.

The temperature sensor sends data every 5 minutes to the MQTT broker, so a 7-minute keepalive interval should be sufficient. This way, no PING messages are necessary, as the broker will assume the sensor has failed if it doesn’t receive a message within that time frame.

On the other hand, the valve needs to send ping messages, since once it subscribes to the temperature topic, it won't communicate with the broker again. A 5-minute keepalive interval is sufficient to guarantee that the valve gets the updates of the temperature needed to compute the average temperature.

We use QoS 0 because the network is reliable, which helps reduce energy consumption. Consequently, the broker won't send PUBACK messages to the temperature sensor every time it receives a PUBLISH message.

For both the valve and the sensor, the first messages exchanged with the broker will be the CONNECT and CONNACK messages.

\subsection{Energy Consumption of the Valve}
To compute the energy $E_V$ used by the valve, we take into account:
\begin{itemize}
    \item the transmission of the CONNECT message ($54B$)
    \item the reception of the CONNACK message ($47B$)
    \item the transmission of the SUBSCRIBE message ($58B$)
    \item the reception of the SUBACK message ($52B$)
    \item the transmission every 5 minutes of the PINGREQ message ($52B$): $60/5=12$ messages per hour
    \item the reception every 5 minutes of the PINGRES message ($48B$)
    \item the reception of the PUBLISH message containing the updated value of the temperature 12 times per hour ($68B$)
    \item the 2 processing per hour of the average temperature (1 every 30 minutes)
\end{itemize}

\begin{equation}
 \begin{aligned}
    E_{V} = E_{TX}*54*8 [nJ] + E_{RX}*47*8 [nJ] &+ \\
    E_{TX}*58*8 [nJ] + E_{RX}*52*8 [nJ] &+ \\
    12*24*(E_{TX}*52*8 + E_{RX}*48*8) &+ \\
    12*24*E_{RX}*68*8 &+ \\
    2*24*E_c [mJ] &= 136.78 [mJ]
 \end{aligned}
\end{equation}

\subsection{Energy Consumption of the Temperature Sensor}
To compute the energy $E_S$ of the temperature sensor, we take into account:
\begin{itemize}
    \item the transmission of the CONNECT message ($54B$)
    \item the reception of the CONNACK message ($47B$)
    \item the transmission of the PUBLISH message containing the updated value of the temperature 12 times per hour ($68B$)
\end{itemize}
\begin{equation}
    E_S = E_{TX}*54*8 [nJ] + E_{RX}*47*8 [nJ] + 12*24*E_{TX}*68*8 [nJ] = 7.88 [mJ]
\end{equation}

\subsection{Total Energy Consumption}
The total energy consumed by the valve and the temperature sensor is given by:
\begin{equation}
    E = E_V + E_T = 144.66 [mJ]
\end{equation}

\chapter{EQ2}
\section{Methodology}
Since computing the average temperature every 30 minutes is the most energy-intensive task, energy consumption could be reduced by offloading this computation to the Raspberry Pi (which is likely connected to a constant power source).

In this setup, the Raspberry Pi would run both the MQTT broker and an MQTT client. The sensor publishes temperature readings to a specific topic (for example sensor/temp), which the Raspberry Pi’s client subscribes to to collect and process the average.

The client running on the Raspberry Pi computes the average temperature every 30 minutes and publishes a control message (OPEN or CLOSE) to another topic (for example valve/control). 

The valve subscribes to this control topic. This way, it no longer performs any computation, but only reacts to commands, significantly reducing its energy usage.

\newpage
\subsection{Energy Consumption of the Valve}
In this new scenario, the energy consumption of the temperature sensor remains the same as before ($7.88$ $mJ$), while the new energy $E_V$ of the valve takes into account:

\begin{itemize}
    \item the transmission of the CONNECT message ($54B$)
    \item the reception of the CONNACK message ($47B$)
    \item the transmission of the SUBSCRIBE message ($58B$)
    \item the reception of the SUBACK message ($52B$)
    \item the transmission every 5 minutes of the PINGREQ message ($52B$): $60/5=12$ messages per hour
    \item the reception every 5 minutes of the PINGRES message ($48B$)
    \item the reception of the PUBLISH message containing the command to either close or open 12 times per hour ($68B$)
\end{itemize}

\begin{equation}
 \begin{aligned}
    E_{V} = E_{TX}*54*8 [nJ] + E_{RX}*47*8 [nJ] &+ \\
    E_{TX}*58*8 [nJ] + E_{RX}*52*8 [nJ] &+ \\
    12*24*(E_{TX}*52*8 + E_{RX}*48*8) &+ \\
    12*24*E_{RX}*68*8 &= 21.58 [mJ]
 \end{aligned}
\end{equation}

\subsection{Total Energy Consumption}
The total energy consumed by the valve and the temperature sensor is given by:
\begin{equation}
    E = E_V + E_T = 29.46 [mJ]
\end{equation}

\end{document}
